%%%%%%%%%%%%%%%%%%%%%%%%%%%%%%%%%%%%%%%%%%%%%%%%%%%%%%%%%%%%%%%%%%%%%%%%%%%%%%%%
% Medium Length Graduate Curriculum Vitae
% LaTeX Template
% Version 1.2 (3/28/15)
%
% This template has been downloaded from:
% http://www.LaTeXTemplates.com
%
% Original author:
% Rensselaer Polytechnic Institute 
% (http://www.rpi.edu/dept/arc/training/latex/resumes/)
%
% Modified by:
% Daniel L Marks <xleafr@gmail.com> 3/28/2015
%
% Important note:
% This template requires the res.cls file to be in the same directory as the
% .tex file. The res.cls file provides the resume style used for structuring the
% document.
%
%%%%%%%%%%%%%%%%%%%%%%%%%%%%%%%%%%%%%%%%%%%%%%%%%%%%%%%%%%%%%%%%%%%%%%%%%%%%%%%%

%-------------------------------------------------------------------------------
%	PACKAGES AND OTHER DOCUMENT CONFIGURATIONS
%-------------------------------------------------------------------------------

%%%%%%%%%%%%%%%%%%%%%%%%%%%%%%%%%%%%%%%%%%%%%%%%%%%%%%%%%%%%%%%%%%%%%%%%%%%%%%%%
% You can have multiple style options the legal options ones are:
%
%   centered:	the name and address are centered at the top of the page 
%				(default)
%
%   line:		the name is the left with a horizontal line then the address to
%				the right
%
%   overlapped:	the section titles overlap the body text (default)
%
%   margin:		the section titles are to the left of the body text
%		
%   11pt:		use 11 point fonts instead of 10 point fonts
%
%   12pt:		use 12 point fonts instead of 10 point fonts
%
%%%%%%%%%%%%%%%%%%%%%%%%%%%%%%%%%%%%%%%%%%%%%%%%%%%%%%%%%%%%%%%%%%%%%%%%%%%%%%%%

\documentclass[margin]{res}  

% XeLaTeX setup
\usepackage{fontspec} % in order to use other fonts
%\setmainfont[Mapping=tex-text]{Minion Pro} % main font
%\setsansfont[Mapping=tex-text]{AkzidenzGrotesk BQ Light} % sans font
%\setmainfont[Mapping=tex-text]{Bell MT} % main font
\setmainfont[Mapping=tex-text]{Arial} % sans font

% Increase text height
\textheight=700pt

\usepackage{enumitem}
\usepackage[colorlinks=false]{hyperref}

\newenvironment{benumerate}[1]{
    \let\oldItem\item
    \def\item{\addtocounter{enumi}{-2}\oldItem}
    \begin{enumerate}[leftmargin=*,noitemsep]
    \setcounter{enumi}{#1}
    \addtocounter{enumi}{1}
}{
    \end{enumerate}
}

\usepackage{xcolor}
\hypersetup{
    colorlinks,
    linkcolor={red!50!black},
    citecolor={blue!50!black},
    urlcolor={blue!80!black}
}

\usepackage{ragged2e}


\begin{document}



%-------------------------------------------------------------------------------
%	NAME AND ADDRESS SECTION
%-------------------------------------------------------------------------------
\name{\sf \Large Han Wang}
\address{\\ \\Website: \url{https://hannnwang.github.io}\\}
%\address{\\ \\Github : \url{https://github.com/hannnwang}}
\address{\\ \\Email : han.wang@ed.ac.uk\\}
% Uncomment to add a third address
%\address{Address 3 line 1\\Address 3 line 2\\Address 3 line 3}
%-------------------------------------------------------------------------------

\begin{resume}

%-------------------------------------------------------------------------------
%	EDUCATION SECTION
%-------------------------------------------------------------------------------
\section{{\normalfont PERSONAL INFORMATION}}
\begin{itemize}[leftmargin=*,noitemsep]
\item[--]{Birthday: June.16, 1993}
\item[--]{Citizenship: Chinese, with Canadian Permanent Residency}
\end{itemize}

\section{{\normalfont RESEARCH INTERESTS}}
\textbf{Physical Oceanography and Fluid Dynamics}
\begin{itemize}[leftmargin=*,noitemsep]
\item[--]{Wave-mean interactions and disentanglement}
\item[--]{Submesoscale ocean dynamics}
%\item[--]{Analysis of observational data}
\item[--]{Statistical fluid mechanics}
\item[--]{Deep learning}
\end{itemize}

\section{{\normalfont EDUCATION}}
2011 - 2015, \textbf{University of Science and Technology of China (USTC)} 
\begin{itemize}[leftmargin=*,noitemsep]
\item[]{B.S. Atmospheric Sciences}
\item[]{Dissertation advisor: Rui Li}
\end{itemize}
\vspace*{-2.5mm}
2015 - 2020, \textbf{Courant Institute, New York University (NYU)} 
\begin{itemize}[leftmargin=*,noitemsep]
\item[]{Ph.D. Atmosphere-Ocean Science and Mathematics}
\item[]{Dissertation advisor: Oliver B{\"u}hler}
\end{itemize}


\section{{\normalfont WORK \\EXPERIENCE}}
2020/09 - 2022/03 (1.5 years), Postdoctoral Researcher
\begin{itemize}[leftmargin=*,noitemsep]
\item[]{Department of Physics, University of Toronto}
\item[]{Supervised by Nicolas Grisouard}
\end{itemize}
2022/04 - present, Postdoctoral Research Associate
\begin{itemize}[leftmargin=*,noitemsep]
\item[]{School of Mathematics, University of Edinburgh}
\item[]{Supervised by  Jacques Vanneste and William R. Young }
\end{itemize}

%\section{{\normalfont TECHNICAL\\SKILLS}}
%\begin{itemize}[leftmargin=*,noitemsep]
%\item[--]{Programming language and computational software : }
%\end{itemize}
%\begin{itemize}[leftmargin=16pt,noitemsep]
%\item[]{Experienced in Python, MATLAB,  Wolfram Mathematica, shell scripting}
%\item[]{Some experience in C, Fortran 90}
%\end{itemize}
%\begin{itemize}[leftmargin=*,noitemsep]
%\item[--]{Editing tools : Adobe Creative Suite, \LaTeX, Microsoft Office}
%\end{itemize}


\section{{\normalfont REFEREED\\ARTICLES}}
\begin{benumerate}{5}
\item \textbf{Wang, H.}, B\^{o}as, A.B.V., Young, W.R. and Vanneste J. (2023). \textit{Scattering of swell by currents}. Journal of Fluid Mechanics, 975.  \href{https://arxiv.org/abs/2305.12163}{[arXiv]}
\item \textbf{Wang, H.}, Grisouard, N., Salehipour, H., Nuz, A., Poon, M., and Ponte, A. L. (2022). \textit{A deep learning approach to extract internal tides scattered by geostrophic turbulence}. Geophysical Research Letters, 49(11), e2022GL099400. \href{https://essopenarchive.org/doi/full/10.1002/essoar.10508849.2}{[ESSOAR]}
\item Khatri, H., Griffies, S. M., Uchida, T., \textbf{Wang, H.},  and Menemenlis, D. (2021). \textit{Role of mixed-layer instabilities in the seasonal evolution of eddy kinetic energy spectra in a global submesoscale permitting simulation.} Geophysical Research Letters, 48(18), e2021GL094777.
\item \textbf{Wang, H.} and B{\"u}hler, O. (2021). \textit{Anisotropic statistics of Lagrangian structure functions and Helmholtz decomposition.} Journal of Physical Oceanography, 51(5), 1375-1393. \href{https://hannnwang.github.io/WangBuhlerJPO21.pdf}{[pdf]}
\item \textbf{Wang, H.} and B{\"u}hler, O. (2020). \textit{Ageostrophic corrections for power spectra and wave–vortex decomposition.} Journal of Fluid Mechanics, 882. \href{https://hannnwang.github.io/AgeoSpectraJFM20.pdf}{[pdf]}
\item[] -- Highlighted in \href{https://www.cambridge.org/core/journals/journal-of-fluid-mechanics/article/untangling-waves-and-vortices-in-the-atmospheric-kinetic-energy-spectra/1BEB1ABC32CD2CFAB99BAFEE4712CD0C}{Focus on Fluids}
\end{benumerate}

\section{{\normalfont RESEARCH\\IN PROGRESS}}
\begin{itemize}

%\href{https://www.youtube.com/watch?v=wzR_IKaSDw8}{[talk]}
\item[--] B\^{o}as, A.B.V., Vanneste J.,  Wang, H., and Young, W.R. \textit{The imprint of weak ocean currents on surface wave significant wave height.}  (in prep. for Journal of Fluid Mechanics.) \href{https://hannnwang.github.io/surfacewaves_WISE.pdf}{[slides]}  \href{https://www.youtube.com/watch?v=migsRGIAB-c}{[talk]}
\item[--] Balwada, D., Wang, H., and Xie, J.-H. \textit{Wave-vortex impact on Gulf of Mexico Kinetic Energy Distribution.} (in prep. for Journal of Physical Oceanography) \href{https://hannnwang.github.io/slide_trr_2023.pdf}{[slides]}  \href{https://www.youtube.com/watch?v=3HQ-iV7y3gI}{[talk]}    
%\item[--]  Grisouard, N., Jeffery U. and Wang, H., \textit{Synergy with surface density observations in a deep learning approach to disentangle balanced flows and internal tides.} \href{https://hannnwang.github.io/slide_Jeff_EWM2023-1.pdf}{[slides]}
%\item[--] Arbic, B.K., Grisouard, N., Guan, L.-X., Thakur, R., and Wang, H.,   \textit{Performance of a deep learning approach to extract incoherent tides in a high-resolution global ocean model.}  
\end{itemize}

\section{{\normalfont TEACHING}}
\begin{itemize}[leftmargin=*,noitemsep]
\item[--]{Substitute lecturer: Industrial Mathematics [2022-2023 SEM1]}
\item[--]{Workshop tutor (University of Edinburgh): Mathematics in Action B (Fluid Dynamics) [2022-2023 SEM2]}
\item[--]{Undergraduate recitation leader  (Courant Institute): Vector Analysis [MATH-UA.0224-001, Spring 2018]}
\item[--]{Graduate course grader  (Courant Institute): Methods of Applied Maths, Applied Stochastic Analysis, Fluid Dynamics}
\end{itemize}
\section{{\normalfont RESEARCH\\MENTORSHIP}}
\begin{itemize}[leftmargin=*,noitemsep]
\item[--]{MSc project supervision : Sole supervisor for Simin Wang on Computational and Applied Mathematics MSc, 2023/06 - 2023/08} 
\item[--]{PhD project supervision: Jeff Uncu, University of Toronto, started in 2022/10}
\item[--]{Undergraduate students: Kerryn Van Rooyen, University of Toronto, 2022/05-2022/08; Lingxiao Guan, University of Michigan, 2021/07-2021/11}
\end{itemize}

\section{{\normalfont SELECTED\\CONFERENCE\\ PRESENTATIONS}}
\begin{itemize}[leftmargin=*,noitemsep]
\item[--]``Imprint of Currents on Surface Waves".  WISE Zoominar on Waves over spatial inhomogeneity, virtual, Oct 2023
%EGU swell presentation
\item[--]``Dynamical insights from frequency-filtered Lagrangian structure functions". TRR 181 Eddy-Wave Meeting, virtual, Feb 2023
\item[--] ``Imprint of ocean currents on significant wave height." 103rd American Meteorological Society Annual Meeting (23rd Conference on Air-Sea Interaction), virtual, Jan 2023
\item[--] “A deep learning approach to extract surface internal tidal signals scattered by geostrophic turbulence.”  Oberwolfach Workshop 2238 - Multiscale Wave-Turbulence Dynamics in the Atmosphere and Ocean, Oberwolfach, Germany, Sep 2022
\item[--] “Internal tidal extraction: challenges from scatterings by vortices, and hopes for a deep learning solution”, Surface Water and Ocean Topography (SWOT) Science Team Meeting, virtual, Jun 2022
%\item[--]  “A deep learning approach to extract surface internal tidal signals scattered by geostrophic turbulence.” IX International Symposium on Stratified Flows, Cambridge, UK, Aug 2022 
\item[--] “Extraction of tidal signals from a machine learning approach”, The challenge of understanding rapidly changing small-scale ocean dynamics: preparation for SWOT, Ocean Sciences Meeting, virtual, Apr 2022
%\item another SWOT stuff I did under the same session with Spencer Jones
%\item[--] “Anisotropic statistics of Lagrangian structure functions and Helmholtz decomposition”, Lagrangian methods for atmosphere and ocean sciences, EGU General Assembly, virtual, Apr 2021 
\item[--] “Generalizing the “BCF14” method to anisotropic cases”. Meeting on eddies and internal waves with TRR Mercator fellows, TRR 181, virtual, Mar 2021
\item[--] “Anisotropic Helmholtz decomposition of Lagrangian Tracer Data”. Poster session for Mesoscale and Submesoscale Ocean Dynamics, 22nd Conference on Atmospheric and Oceanic Fluid Dynamics, Maine, USA, Jun 2019
\item[--] “Wave-Vortex Decomposition of 1D ship-track data with weak nonlinearity in the balanced flow”. Poster session for Theoretical Advances in AOFD, 21st Conference on Atmospheric and Oceanic Fluid Dynamics, Oregon, USA Jun 2017
\end{itemize}
%-------------------------------------------------------------------------------
\section{{\normalfont HONORS}}\begin{itemize}[leftmargin=*,noitemsep]
\item[--]{MacCracken Fellowship, five-year graduate student award at NYU, 2015-2020}
%\item[--]{\textit{Zhao Jiuzhang} Sci-Tech Elite Class of Modern Earth and Space Science, USTC, 2013-2015}
%\item[--]{Receieved fellowship for undergraduate summer research overseas. Funded by China Scholarship Council}
\end{itemize}


%\section{{\normalfont EDUCATIONAL \\EXPERIENCE}}
%\begin{itemize}[leftmargin=*,noitemsep]
%\item[--] {2017/07--2017/08, Participant of Les Houches Summer School on Fundamental Aspects of Turbulent Flows in Climate Dynamics, Les Houches, France; funded by organizer}
%\item[--]{2014/06--2014/08, Visiting scholar at University of Michigan, Ann Arbor, USA. Conducted pedagogical experiments on CMIP5 Models, advised by Xianglei Huang; funded by China Scholarship Council }
%\item[--]{2013/09--2014/01, Exchange student in Physics Department, National Tsing Hua University(NTHU), Taiwan; }funded by NTHU and USTC 
%\end{itemize}
 
 
\section{{\normalfont SERVICE}}
\begin{itemize}[leftmargin=*,noitemsep]
\item[--]{Journal Referee: Geophysical Research Letters, Journal of Fluid Mechanics, Journal of Physical Oceanography, Quarterly Journal of the Royal Meteorological Society, Journal of Advances in Modeling Earth Systems, Journal of Atmospheric and Oceanic Technology}
%\item[--]{Student host for guest colloquium lunches (2019/09--2020/09)}
\item[--]{Organizer: 2-week workshop on Theoretical and Practical Perspectives in Geophysical Fluid Dynamics to take place in ICTS, India in May 2024, funded by ICTS and TRR181. (Participated in workshop funding application and various organizational tasks.);  various local seminars at Courant Institute and University of Edinburgh.}
\item[--]{Outreach: Courant cSplash lecturer (2019/04); collaboration with ICMS visiting fellow in music (2022/10-2023/10), with an upcoming presentation at an outreach session in Ocean Sciences meeting (2024/02).}
%\item[--]{Social Volunteer and outreach: Courant cSplash (2019/04), NYC H2O (2018-2020), 
%Sutton Trust Summer School Info Station (2023)}
%\item[--]{Frequent volunteer of NYC H2O events including beach cleanups and landscapings}
%\item[--]{Volunteer photographer at CAOS}
\end{itemize}
\end{resume}

\pagenumbering{gobble}

\begin{flushright}
Updated in November, 2023
\end{flushright}

%\let\thefootnote\relax\footnotetext{Updated in July, 2023}
%\let\cleardoublepage\clearpage
%\(\)

\end{document}

